\documentclass[tikz]{standalone}
\begin{document}
\usetikzlibrary{math}
\usetikzlibrary{calc}
\begin{tikzpicture}
% string locations
\coordinate (s6) at (0,+0.25);
\coordinate (s5) at (0,+0.75);
\coordinate (s4) at ($(s5)!-1!(s6)$);
\coordinate (s3) at ($(s4)!-1!(s5)$);
\coordinate (s2) at ($(s3)!-1!(s4)$);
\coordinate (s1) at ($(s2)!-1!(s3)$);
% fret locations
\tikzmath{\mihs = -pow(2,-1/12);}
\coordinate (f0)  at (0,0);
\coordinate (f1)  at (1,0);
\coordinate (f2)  at ($(f1) !\mihs!(f0)$);
\coordinate (f3)  at ($(f2) !\mihs!(f1)$);
\coordinate (f4)  at ($(f3) !\mihs!(f2)$);
\coordinate (f5)  at ($(f4) !\mihs!(f3)$);
\coordinate (f6)  at ($(f5) !\mihs!(f4)$);
\coordinate (f7)  at ($(f6) !\mihs!(f5)$);
\coordinate (f8)  at ($(f7) !\mihs!(f6)$);
\coordinate (f9)  at ($(f8) !\mihs!(f7)$);
\coordinate (f10) at ($(f9) !\mihs!(f8)$);
\coordinate (f11) at ($(f10)!\mihs!(f9)$);
\coordinate (f12) at ($(f11)!\mihs!(f10)$);
\coordinate (f13) at ($(f12)!\mihs!(f11)$);
% use this path to extend bounding box to margins
\coordinate (top) at ($(s1)!-0.5!(s2)$);
\coordinate (bot) at ($(s6)!-0.5!(s5)$);
\coordinate (right) at ($(f13)!-0.5!(f12)$);
\coordinate (left) at ($(f0)!-0.5!(f1)$);
\path (top -| right) coordinate (UR) -- (bot -| left) coordinate (LL);
% draw frets
\draw[line width=3pt] (f0  |- s1) -- (f0 |- s6);
\draw[thick,gray]     (f1  |- s1) -- (f1 |- s6);
\draw[thick,gray]     (f2  |- s1) -- (f2 |- s6);
\draw[thick,gray]     (f3  |- s1) -- (f3 |- s6);
\draw[thick,gray]     (f4  |- s1) -- (f4 |- s6);
\draw[thick,gray]     (f5  |- s1) -- (f5 |- s6);
\draw[thick,gray]     (f6  |- s1) -- (f6 |- s6);
\draw[thick,gray]     (f7  |- s1) -- (f7 |- s6);
\draw[thick,gray]     (f8  |- s1) -- (f8 |- s6);
\draw[thick,gray]     (f9  |- s1) -- (f9 |- s6);
\draw[thick,gray]     (f10 |- s1) -- (f10 |- s6);
\draw[thick,gray]     (f11 |- s1) -- (f11 |- s6);
\draw[thick,gray]     (f12 |- s1) -- (f12 |- s6);
\draw[thick,gray]     (f13 |- s1) -- (f13 |- s6);
% draw strings
\draw[line width=0.3pt,gray] (s1 -| f0) -- (s1 -| UR);
\draw[line width=0.3pt,gray] (s2 -| f0) -- (s2 -| UR);
\draw[line width=0.6pt,gray] (s3 -| f0) -- (s3 -| UR);
\draw[line width=0.6pt,gray] (s4 -| f0) -- (s4 -| UR);
\draw[line width=1.2pt,gray] (s5 -| f0) -- (s5 -| UR);
\draw[line width=1.5pt,gray] (s6 -| f0) -- (s6 -| UR);
% fret midway points
\coordinate (f0m) at ($(f0)!-0.25!(f1)$);
\coordinate (f1m) at ($(f0)!0.5!(f1)$);
\coordinate (f2m) at ($(f1)!0.5!(f2)$);
\coordinate (f3m) at ($(f2)!0.5!(f3)$);
\coordinate (f4m) at ($(f3)!0.5!(f4)$);
\coordinate (f5m) at ($(f4)!0.5!(f5)$);
\coordinate (f6m) at ($(f5)!0.5!(f6)$);
\coordinate (f7m) at ($(f6)!0.5!(f7)$);
\coordinate (f8m) at ($(f7)!0.5!(f8)$);
\coordinate (f9m) at ($(f8)!0.5!(f9)$);
\coordinate (f10m) at ($(f9)!0.5!(f10)$);
\coordinate (f11m) at ($(f10)!0.5!(f11)$);
\coordinate (f12m) at ($(f11)!0.5!(f12)$);
\coordinate (f13m) at ($(f12)!0.5!(f13)$);
% number the frets
\foreach \i in {0,...,13}{
  \path (f\i) node {\i};
}
% draw dots
\coordinate (s2h3) at ($(s2)!0.5!(s3)$);
\coordinate (s3h4) at ($(s3)!0.5!(s4)$);
\coordinate (s4h5) at ($(s4)!0.5!(s5)$);
\draw [gray] (f5m |- s3h4) circle (4pt);
\draw [gray] (f7m |- s3h4) circle (4pt);
% \draw [gray] (f9m |- s3h4) circle (4pt);
\draw [gray] (f12m |- s2h3) circle (4pt);
\draw [gray] (f12m |- s4h5) circle (4pt);
% draw notes of interest
% C major scale, one octave, 4th string root
\draw [red] (f10m |- s4) circle (5pt) node {2};
\draw       (f12m |- s4) circle (5pt) node {4};
\draw       (f9m  |- s3) circle (5pt) node {1};
\draw       (f10m |- s3) circle (5pt) node {2};
\draw       (f12m |- s3) circle (5pt) node {4};
\draw       (f10m |- s2) circle (5pt) node {1};
\draw       (f12m |- s2) circle (5pt) node {3};
\draw [red] (f13m |- s2) circle (5pt) node {4};
% C major scale, one octave, 3rd string root
\draw [red] (f5m  |- s3) circle (5pt) node {1};
\draw       (f7m  |- s3) circle (5pt) node {3};
\draw       (f5m  |- s2) circle (5pt) node {1};
\draw       (f6m  |- s2) circle (5pt) node {2};
\draw       (f8m  |- s2) circle (5pt) node {4};
\draw       (f5m  |- s1) circle (5pt) node {1};
\draw       (f7m  |- s1) circle (5pt) node {3};
\draw [red] (f8m  |- s1) circle (5pt) node {4};
\end{tikzpicture}
\end{document}
